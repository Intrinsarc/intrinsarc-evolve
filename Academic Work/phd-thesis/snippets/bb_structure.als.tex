\lstset{frame=tb, aboveskip=12pt, belowskip=-3pt, breaklines=true, tabsize=2, mathescape=true}
\begin{lstlisting}[caption={bb\_structure.als}, numbers=left]
module bb_structure

open base_deltas[PartID, Part] as Parts
open base_deltas[PortID, Port] as Ports
open base_deltas[ConnectorID, Connector] as Connectors
open base_deltas[AttributeID, Attribute] as Attributes
open base_deltas[OperationID, Operation] as Operations
open base_deltas[InterfaceImplementationID, InterfaceImplementation] as InterfaceImplementation
open base_deltas[ComponentImplementationID, ComponentImplementation] as ComponentImplementation
open base_deltas[PrimitiveTypeImplementationID, PrimitiveTypeImplementation] as PrimitiveTypeImplementation
open base_deltas[LinkID, Link] as Links

sig Component extends Element
{
  myParts: lone Parts/Deltas,
  myPorts: lone Ports/Deltas,
  myConnectors: lone Connectors/Deltas,
  myAttributes: lone Attributes/Deltas,
  myCImplementation: lone ComponentImplementation/Deltas,
  myLinks: lone Links/Deltas,
  
  -- the final result, after taking replacement + resemblance into account
  iDParts: PartID -> Part -> Stratum,
  -- composite or leaf?
  isComposite: set Stratum,
  parts: Part -> Stratum,
  ports: Port -> Stratum,
  connectors: Connector -> Stratum,
  attributes: Attribute -> Stratum,
  cimplementation: ComponentImplementation -> Stratum,
  links: Link -> Stratum,
  
  -- the internal links, used for port type inferencing
  inferredLinks: Port -> Port -> Stratum
}
{
  -- rule: COMPONENT_RESEMBLANCE -- for components
  replaces + resembles in Component
  -- propagate up the objects from the delta into the sig, to make it more convenient
  parts = myParts.objects
  ports = myPorts.objects
  connectors = myConnectors.objects
  attributes = myAttributes.objects
  cimplementation = myCImplementation.objects
  links = myLinks.objects

  -- form idParts
  iDParts = {n: PartID, p: Part, s: Stratum | s -> n -> p in myParts.objects_e}
}

-- ensure each delta is composed by only one component
fact
{
  all p: Parts/Deltas | one myParts.p
  all p: Ports/Deltas | one myPorts.p
  all c: Connectors/Deltas | one myConnectors.c
  all a: Attributes/Deltas | one myAttributes.a
  all i: ComponentImplementation/Deltas | one myCImplementation.i
  all l: Links/Deltas | one myLinks.l
}

sig PrimitiveType extends Element
{
  myTImplementation: lone PrimitiveTypeImplementation/Deltas,
  -- the expanded elements
  timplementation: PrimitiveTypeImplementation -> Stratum
}
{
  replaces + resembles in PrimitiveType
  -- propagate up the objects from the delta into the sig, to make it more convenient
  timplementation = myTImplementation.objects
}
-- ensure each delta is composed by only one primitive type
fact
{
  all t: PrimitiveTypeImplementation/Deltas | one myTImplementation.t
}

sig Interface extends Element
{
  -- the deltas
  myOperations: lone Operations/Deltas,
  myIImplementation: lone InterfaceImplementation/Deltas,
  -- the expanded elements
  operations: Operation -> Stratum,  
  iimplementation: InterfaceImplementation -> Stratum,  
  superTypes: Interface -> Stratum
}
{
  -- for interfaces
  replaces + resembles in Interface
  -- propagate up the objects from the delta into the sig, to make it more convenient
  operations = myOperations.objects  
  iimplementation = myIImplementation.objects
}
-- ensure each delta is composed by only one interface
fact
{
  all p: Operations/Deltas | one myOperations.p
  all i: InterfaceImplementation/Deltas | one myIImplementation.i
}


-- each artifact must have a id, so it can be replaced or deleted
sig PartID, PortID, ConnectorID, AttributeID, OperationID, InterfaceImplementationID, ComponentImplementationID, PrimitiveTypeImplementationID, LinkID {}

sig Part
{
  -- rule: WF_PART_TYPE -- each part must have a type
  partType: Component,
  -- remap a port from this part onto the port of a part that we are replacing
  -- (new port -> old, replaced port)
  portRemap: PortID lone -> lone PortID,
  portMap: Stratum -> PortID lone -> lone Port,

  -- the values of the attributes are set in the part   (child id -> parent id)
  -- although they don't have to be set if we want to take the default
  attributeValues: AttributeID -> lone AttributeValue,
  -- do we alias a parent attribute?
  attributeAliases: AttributeID -> lone AttributeID,
  -- or do we simply copy a parent attribute, but retain our own state?
  attributeCopyValues: AttributeID -> lone AttributeID,

  -- derived state -- the parts that the connectors link to
  linkedToParts: Part -> Stratum -> Component,
  -- derived state -- any componts that the connectors link to
  linkedToOutside: Stratum -> Component
}

abstract sig Index {}
one sig Zero, One, Two, Three extends Index {}

pred isContiguousFromZero(indices: set Index)
{
  indices = indices.*(Three->Two + Two->One + One->Zero)
}

sig Port
{
  -- set values are what the user has explicitly set
  setProvided, setRequired: set Interface,
  -- provided and required are inferred
  provided, required: Interface -> Stratum -> Component,
  mandatory, optional: set Index
}
{
  -- mandatory indices start at 0, optional start from mandatory end, no overlap
  -- all contiguous and must have some indices
  -- rule: PORT_MULTIPLICITY
  isContiguousFromZero[mandatory] and
    isContiguousFromZero[mandatory + optional]
  no mandatory & optional      -- no overlap
  some mandatory + optional    -- but must have some indices
}

sig Connector
{
    -- require 2 ends
    ends: set ConnectorEnd
}
{
  -- ensure 2 connector ends using a trick felix taught me
  some disj end1, end2: ConnectorEnd | ends = end1 + end2
    all end: ends |
        end.otherEnd = ends - end
}

abstract sig ConnectorEnd
{
    portID: PortID,
    port: Port -> Stratum -> Component,
    index: Index,
    otherEnd: ConnectorEnd
}
{
    -- an end is owned by one connector
    one ends.this
}

sig ComponentConnectorEnd extends ConnectorEnd
{
}

sig PartConnectorEnd extends ConnectorEnd
{
    partID: PartID,
    cpart: Part -> Stratum -> Component
}


sig Attribute
{
  -- rule: WF_ATTRIBUTE_TYPE -- an attribute must have a type
  attributeType: PrimitiveType,
  defaultValue: lone AttributeValue
}
{
  -- rule: WF_ATTRIBUTE_DEFAULT
  some defaultValue =>
    defaultValue.valueType = attributeType
}

sig AttributeValue
{
  valueType: PrimitiveType
}

sig Operation
{
  -- this identifies the impelementation id and signature
}

sig InterfaceImplementation
{
  -- this identifies the interface implementation clas or no s.dependsOns...
}

sig ComponentImplementation
{
  -- this identifies the component implementation class...
}

sig PrimitiveTypeImplementation
{
    -- this indentified the implementation of a primitive type
}

-- links are used for port inference
-- a bit like a connector, but multiplicity and optionality don't count
sig Link
{
  linkEnds: PortID -> PortID
}
{
  lone linkEnds
}

abstract sig LinkEnd
{
  linkPortID: PortID,
  linkError: Stratum -> Component
  -- the internal interfaces are the interfaces presented inside the component content area
  --   for a port, it is the interfaces seen internally (opposite)
  --   for a port instance, it is the interfaces seen externally (same)
}

sig ComponentLinkEnd extends LinkEnd
{
}

sig PartLinkEnd extends LinkEnd
{
  linkPartID: PartID
}

-- translate from port id to component link end -- guaranteed to be 1 per id
fun getComponentLinkEnd(id: PortID): one ComponentLinkEnd
{
  { end: ComponentLinkEnd | end.linkPortID = id }
}

-- translate from a port/part to a part link end -- guaranteed to be 1 per pair
fun getPartLinkEnd(portID: PortID, partID: PartID): PartLinkEnd
{
  { end: PartLinkEnd | end.linkPortID = portID and end.linkPartID = partID }
}

fun ComponentLinkEnd::getPort(s: Stratum, c: Component): one Port
{
  c.myPorts.objects_e[s][this.linkPortID]
}

fun PartLinkEnd::getPortInstance(s: Stratum, c: Component): Port -> Part
{
  let
    cpart = c.myParts.objects_e[s][this.linkPartID],
    cport = cpart.partType.myPorts.objects_e[s][this.linkPortID] |
  cport -> cpart
}

-- get the port of a component connector
fun ComponentConnectorEnd::getPort(s: Stratum, c: Component): lone Port
{
    c.myPorts.objects_e[s][this.portID]
}

-- should return only 1 Port, unless the component is invalid. NOTE: the component owns the part
fun PartConnectorEnd::getPortInstance(s: Stratum, c: Component): Port -> Part
{
    let
        ppart = c.myParts.objects_e[s][this.partID],
        port = ppart.portMap[s][this.portID] |
    port -> ppart
}

fun PartLinkEnd::getPortInstanceRequired(s: Stratum, c: Component): set Interface
{
  let portPart = this.getPortInstance[s, c],
    pport = dom[portPart],
    ppartType = ran[portPart].partType
  {
    pport.required.ppartType.s
  }
}

fun PartLinkEnd::getPortInstanceProvided(s: Stratum, c: Component): set Interface
{
  let portPart = this.getPortInstance[s, c],
    pport = dom[portPart],
    ppartType = ran[portPart].partType
  {
    pport.provided.ppartType.s
  }
}
\end{lstlisting}
